\chapter{Software}
In wat volgt, wordt het ontwerpproces van de software uitvoerig besproken en wordt een verantwoording gegeven van de ontwerpskeuzes.

\section{Verbinding tussen de Ultra Wide Band Location Anchors en de Decawave} \label{sec:uwb_decawave}
Ultra Wide Band (UWB) \cite{alarifi2016ultra}.

\section{Verbinding tussen de Decawave en de Raspberry Pi} \label{sec:decawave_raspberry}
I\textsuperscript{2}C / TWI

\section{Verbinding tussen de Raspberry Pi en de drone} \label{sec:raspberry_drone}
De drone heeft een eigen wifi-netwerk met ESSID adrone2\_xxx en geeft zichzelf vaak het IP-adres 192.168.1.1.
Als de Raspberry Pi Zero W verbindt met het netwerk van de drone, krijgen het een IP-adres tussen 192.168.1.2 en 192.168.1.5 (met de grenzen inbegrepen) toegekend.
Indien de drone een ander IP-adres aan zichzelf toegekend heeft, zullen de gebruikers één van de 4 volgende adressen toegekend krijgen.
Het besturen van de drone gebeurd door het versturen van \textit{AT commands} op UDP poort 5556.
De frequentie waarmee de commando's moeten doorgestuurd worden ligt rond de \SI{30}{\Hz} of met een tussenperiode van ongeveer \SI{30}{\ms}, om de gebruiker een ervaring van voldoende hoge kwaliteit te voorzien.
Wanneer er tussen 2 opeenvolgende commando's meer dan \SI{2}{\s} zitten, zal de AR Drone denken dat de verbinding verbroken is.\\
Informatie over de drone (zoals status, positie, snelheid, snelheid van de rotoren, ...) wordt naar de gebruiker gestuurd op UDP poort 5554.
De frequentie waarmee deze \textit{navdata} wordt verstuurd ligt tussen de \SI{15}{\Hz} (in demo mode) en \SI{200}{\Hz} in full (debug) mode.\\
Om belangrijke data, zoals informatie voor de configuratie, te versturen maakt men geen gebruik van UDP, maar van TCP.
Dit gebeurd via de \textit{control port} 5559.\\
\\
\textit{Syntax van AT commands en navdata is terug te vinden in hoofdstuk 6 van ARDrone\_Developer\_Guide.pdf (Project $\to$ ARDrone\_SDK\_2\_0\_1 $\to$ Docs)!}\\
\textit{Configuratie van de drone is terug te vinden in hoofdstuk 8 van ARDrone\_Developer\_Guide.pdf (Project $\to$ ARDrone\_SDK\_2\_0\_1 $\to$ Docs)!}

\section{Verbinding tussen de Raspberry Pi en de server} \label{sec:raspberry_server}


\section{Full-mesh} \label{sec:full_mesh}
