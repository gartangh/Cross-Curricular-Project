\begin{abstract}
Voorraadbeheer in een magazijn gebeurt traditioneel door werknemers die op een vorklift handmatig barcodes inscannen.
Dit is een zeer langzame, dure en gevaarlijke aanpak.
Een kosteneffectievere oplossing maakt gebruik van automatisch aangestuurde drones die met een hun camera barcodes kunnen inscannen.
Het doel van dit vakoverschrijdend project is om een basis te leggen voor dit systeem, door commercieel beschikbare drones te voorzien van een controller.\\

A.d.h.v. Pozyx location anchors die op gekende plaatsen hangen, een Pozyx tag die gemonteerd is op de drone en Mosquitto server kan de locatie van de drone nauwkeurig bepaald worden.
Vervolgens kan de drone aangestuurd worden met een beschikbare Node.js library.\\

Het belangrijkste resultaat is de implementatie van het volledige systeem.
Hierbij vinden er locatie-updates plaats met een frequentie van \SI{27.5}{\Hz} en wordt de drone aangestuurd met een frequentie van \SI{10}{\Hz}.\\

\keywords{Drone, Ultra-wideband (UWB)}
\end{abstract}