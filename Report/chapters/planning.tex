\chapter{Planning}

\section{Initi\"ele planning} \label{sec:initiele_planning}
De initi\"ele planning is terug te vinden op figuur \ref{fig:initiele_planning}.
\begin{figure}[p]
\centering
\begin{ganttchart}[vgrid, y unit chart=0.75cm, bar/.append style={fill=White, rounded corners=2pt}, milestone/.append style={fill=White}]{1}{13}
	\gantttitlelist{1,...,13}{1}\\

	\ganttgroup{Hardware}{1}{5}\\
	\ganttbar{Uitzoeken en bestellen}{1}{2}\\
	\ganttbar[bar/.append style={fill=JungleGreen}]{Raspberry en DecaWave verbinden}{4}{5}\\
	\ganttbar[bar/.append style={fill=RedOrange}]{Power supply}{5}{5}\\
	\ganttbar[bar/.append style={fill=RedOrange}]{Controller aan drone bevestigen}{5}{5}\\

	\ganttgroup{Software}{3}{12}\\
	\ganttbar[bar/.append style={fill=Yellow}]{UWB}{3}{4}\\
	\ganttbar[bar/.append style={fill=Fuchsia}]{Connectie met server}{5}{6}\\
	\ganttbar[bar/.append style={fill=Fuchsia}]{Aansturing drone}{4}{9}\\
	\ganttbar{Full-mesh}{10}{12}\\

	\ganttgroup{Presentaties en verslag}{1}{13}\\
	\ganttmilestone[milestone/.append style={fill=JungleGreen}]{Presentatie 1}{4}\\
	\ganttlinkedmilestone[milestone/.append style={fill=Cyan}]{Presentatie 2}{8}\\
	\ganttlinkedmilestone{Verslag}{12}\\
	\ganttlinkedmilestone{Eindpresentatie}{13}
\end{ganttchart}
\caption[Gantt chart van de initi\"ele planning.]{Gantt chart van de initi\"ele planning.}
\label{fig:initiele_planning}
\end{figure}

\section{Taakverdeling} \label{sec:taakverdeling}
Wie doet wat? Dit kunt u terug vinden in tabel \ref{tab:taakverdeling}.
\begin{table}[p]
\centering
\begin{tabular}{ |c|c|c| } \hline
Wat? & Wie? & Kleur \\ [.5ex] \hline\hline
Uitzoeken en bestellen & Iedereen & WIT \\ \hline
Rasberry en Pozyx verbinden & Bram en Garben & \textcolor{JungleGreen}{JUNGLEGROEN} \\ \hline
Power supply & Garben & \textcolor{RedOrange}{ROODORANJE} \\ \hline
Controller houder & Garben & \textcolor{RedOrange}{ROODORANJE} \\ \hline
UWB & Bram & \textcolor{Yellow}{GEEL} \\ \hline
Connectie met server & Iedereen & WIT \\ \hline
Aansturing drone & Robbe en Garben & \textcolor{Fuchsia}{FUCHSIA} \\ \hline
Visualisatie & Xavier & \textcolor{Sepia}{SEPIA} \\ \hline
Presentatie 1 & Bram en Garben & \textcolor{JungleGreen}{JUNGLEGROEN} \\ \hline
Presentatie 2 & Robbe en Xavier & \textcolor{Cyan}{CYAAN} \\ \hline
Verslag & Iedereen & WIT \\ \hline
Eindpresentatie & Iedereen & WIT \\ \hline
\end{tabular}
\caption[Taakverdeling]{Taakverdeling.}
\label{tab:taakverdeling}
\end{table}

\section{Finale planning} \label{sec:finale_planning}
De finale planning is terug te vinden op figuur \ref{fig:finale_planning}.
\begin{figure}[p]
\centering
\begin{ganttchart}[vgrid, y unit chart=0.75cm, bar/.append style={fill=White, rounded corners=2pt}, milestone/.append style={fill=White}]{1}{13}
	\gantttitlelist{1,...,13}{1}\\
	
	\ganttgroup{Hardware}{1}{11}\\
	\ganttbar{Uitzoeken en bestellen}{1}{3}\\
	\ganttbar[bar/.append style={fill=JungleGreen}]{Raspberry en Pozyx verbinden}{4}{5}\\
	\ganttbar[bar/.append style={fill=RedOrange}]{Power supply}{8}{8}\\
	\ganttbar[bar/.append style={fill=RedOrange}]{Controller houder}{10}{11}\\
	
	\ganttgroup{Software}{3}{12}\\
	\ganttbar[bar/.append style={fill=Yellow}]{UWB}{3}{7}\\
	\ganttbar[bar/.append style={fill=White}]{Connectie met server}{5}{8}\\
	\ganttbar[bar/.append style={fill=Fuchsia}]{Aansturing drone}{4}{12}\\
	\ganttbar[bar/.append style={fill=Sepia}]{Visualizatie}{6}{10}\\
	
	\ganttgroup{Presentaties en verslag}{1}{13}\\
	\ganttmilestone[milestone/.append style={fill=JungleGreen}]{Presentatie 1}{4}\\
	\ganttlinkedmilestone[milestone/.append style={fill=Cyan}]{Presentatie 2}{8}\\

	\ganttlinkedmilestone{Verslag}{12}\\
	\ganttlinkedmilestone{Eindpresentatie}{13}
\end{ganttchart}
\caption[Gantt chart van de finale planning.]{Gantt chart van de finale planning.}
\label{fig:finale_planning}
\end{figure}\\

De grootste verschillen tussen de initi\"ele planning en de finale planning zijn dat de visualisatie er bij is gekomen en dat het hardware gedeelte later ge\"eindigd is.\\

De motivatie van de belangrijkste wijzigingen zijn dat het visuele gedeelte enorm uitgebreid is ten opzichte van wat initi\"eel gepland werd. Zo worden nu meerdere drones live ondersteund in Unity en kun je die volgen met verscheidene camera's. Ook de paden kunnen aangemaakt, bewerkt en verstuurd worden vanuit Unity.\\

De reden waarom het hardware gedeelte langer duurde dan oorspronkelijk ingepland werd, is vrij simpel te verklaren.\\
In het begin werden enkele afwegingen gemaakt (vb. tussen verschillende drones en de DecaWave tegenover de Pozyx tag), die pas na enkele weken vastgelegd werden.\\
Om de kosten te drukken werden kabels, batterijen, ... besteld uit China. Het nadeel daarvan is dat onderdelen enkele weken onderweg kunnen zijn, waardoor die pas laat voorhanden waren. De code kon ondertussen al geschreven worden, maar nog niet uitgetest op de hardware.\\
De houder voor de controller werd pas in de laatste 3 weken in 3D getekend en geprint.
De reden waarom er pas in de laatste weken voor de controller houder gekeken werd is dat de prioriteit op andere delen lag. Indien er tijd over zou zijn, kon er gekeken worden om een esthetischere houder te cre\"eren dan wat wit plakband.