\chapter{Planning}

\section{Initi\"ele planning} \label{sec:initiele_planning}
De initi\"ele planning is terug te vinden op figuur \ref{fig:initiele_planning}.
\begin{figure}[p]
\centering
\begin{ganttchart}[vgrid, y unit chart=0.75cm, bar/.append style={fill=White, rounded corners=2pt}, milestone/.append style={fill=White}]{1}{13}
	\gantttitlelist{1,...,13}{1}\\

	\ganttgroup{Hardware}{1}{5}\\
	\ganttbar[name=uitzoeken_bestellen]{Uitzoeken en bestellen}{1}{2}\\
	\ganttbar[name=raspberry_decawave, bar/.append style={fill=JungleGreen}]{Raspberry en Decawave verbinden}{4}{4}\\
	\ganttbar[name=power_supply, bar/.append style={fill=RedOrange}]{Power supply}{5}{5}\\
	\ganttbar[name=controller_drone, bar/.append style={fill=RedOrange}]{Controller aan drone bevestigen}{5}{5}\\

	\ganttgroup{Software}{3}{12}\\
	\ganttbar[name=uwb, bar/.append style={fill=Yellow}]{UWB}{3}{4}\\
	\ganttlinkedbar[name=i2c_twi, bar/.append style={fill=Yellow}]{I\textsuperscript{2}C / TWI}{6}{8}\\
	\ganttbar[name=aansturen_drone, bar/.append style={fill=Fuchsia}]{Aansturing drone}{3}{5}\\
	\ganttlinkedbar[name=connectie_server, bar/.append style={fill=Fuchsia}]{Connectie met server}{6}{8}\\
	\ganttbar[name=full_mesh, bar/.append style={fill=Sepia}]{Full-mesh}{9}{12}\\
	\ganttbar[name=collision_detection, bar/.append style={fill=Salmon}]{Collision detection}{9}{12}\\

	\ganttgroup{Presentaties en verslag}{1}{13}\\
	\ganttmilestone[name=presentatie_1, milestone/.append style={fill=JungleGreen}]{Presentatie 1}{4}\\
	\ganttlinkedmilestone[name=presentatie_2, milestone/.append style={fill=RedOrange}]{Presentatie 2}{8}\\
	\ganttlinkedmilestone[name=verslag]{Verslag}{12}\\
	\ganttlinkedmilestone[name=eindpresentatie]{Eindpresentatie}{13}

	\ganttlink[link mid=.1]{uitzoeken_bestellen}{uwb}
	\ganttlink[link mid=.0714]{uitzoeken_bestellen}{aansturen_drone}
\end{ganttchart}
\caption[Gantt chart van de initi\"ele planning.]{Gantt chart van de initi\"ele planning.}
\label{fig:initiele_planning}
\end{figure}

\section{Taakverdeling} \label{sec:taakverdeling}
Wie doet wat? Dit kun je terug vinden in tabel \ref{tab:taakverdeling}.
\begin{table}[p]
\centering
\begin{tabular}{ |c|c| } \hline
Wat? & Wie? \\ [.5ex] \hline\hline
Uitzoeken en bestellen & Iedereen (WIT) \\ \hline
Raspberry en Decawave verbinden & Bram en Garben (JUNGLEGROEN) \\ \hline
Power supply & Robbe en Xavier (ROODORANJE) \\ \hline
Controller aan drone bevestigen & Robbe en Xavier (ROODORANJE) \\ \hline
UWB & Bram en Xavier (GEEL) \\ \hline
I\textsuperscript{2}C / TWI & Bram en Xavier (GEEL) \\ \hline
Aansturing drone & Garben en Robbe (FUCHSIA) \\ \hline
Connectie met server & Garben en Robbe (FUCHSIA) \\ \hline
Full-mesh &  Garben en Xavier (SEPIA) \\ \hline
Collision detection &  Bram en Robbe (ZALM) \\ \hline
Presentatie 1 & Bram en Garben (JUNGLEGROEN) \\ \hline
Presentatie 2 & Robbe en Xavier (ROODORANJE) \\ \hline
Verslag & Iedereen (WIT) \\ \hline
Eindpresentatie & Iedereen (WIT) \\ \hline
\end{tabular}
\caption[Taakverdeling]{Taakverdeling.}
\label{tab:taakverdeling}
\end{table}

\section{Finale planning} \label{sec:finale_planning}
De finale planning is terug te vinden op figuur \ref{fig:finale_planning}.
\begin{figure}[p]
\centering
\begin{ganttchart}[vgrid, y unit chart=0.75cm, bar/.append style={fill=white, rounded corners=2pt}]{1}{13}
	\gantttitlelist{1,...,13}{1}\\


\end{ganttchart}
\caption[Gantt chart van de finale planning.]{Gantt chart van de finale planning.}
\label{fig:finale_planning}
\end{figure}\\
De grootste verschillen tussen de initi\"ele planning en de finale planning zijn ...\\
De motivatie van de belangrijkste wijzigingen zijn ...