\chapter{Resultaten}
Het belangrijkste resultaat is natuurlijk de implementatie van het volledige drone systeem. Hierbij vinden er locatie-updates plaats met een frequentie van \SI{27.5}{\Hz} en wordt de drone aangestuurd met een frequentie van \SI{10}{\Hz}.\\

Tijdens de proefopstelling is er gebruik gemaakt van vier Pozyx ankers voor de locatiebepaling.
Het doel was dat de drone een op voorhand gedefiniëerd pad vloog, weergegeven met de witte lijnen op figuur \ref{fig:Opstelling}.
Op de figuren \ref{fig:SuccesfullFlight1} en \ref{fig:SuccesfullFlight2} staan de waypoints en de effectief gevlogen paden van twee verschillende testvluchten gevisualiseerd.
De waypoints (rode plustekens) komen overeen met de hoekpunten van de rechthoek voorgesteld op figuur \ref{fig:Opstelling}.
Rond elk waypoint is een cirkel getekend die de grootte van het waypoint weergeeft, hier \SI{150}{\mm}.
Wanneer de drone binnen dit bereik valt, heeft hij het waypoint bereikt.
De stippellijnen op komen overeen met de tegels van de ruimte, deze zijn \SI{600}{\mm} $\times$ \SI{600}{\mm}.
De drone probeert de waypoints de doorlopen in wijzerzin, met het linkerbovenhoek als startpunt. 
\begin{figure}[p]
	\centering
	\includegraphics[width=\textwidth]{Opstelling}
	\caption[Opstelling testvluchten]{Testopstelling voor aansturen van de drone.}
	\label{fig:Opstelling}
\end{figure}

\begin{figure}[p]	
	\centering
	\includegraphics[width=\textwidth]{SuccesfullFlight1}
	\caption[Testvlucht 1]{Testvlucht van de drone waarbij hij langs de vier waypoints passeert.}
	\label{fig:SuccesfullFlight1}
\end{figure}
	
\begin{figure}[p]
	\centering
	\includegraphics[width=\textwidth]{SuccesfullFlight2}
	\caption[Testvlucht 2]{Testvlucht 1.}
	\label{fig:SuccesfullFlight2}
\end{figure}

\section{Testen} \label{sec:test}
Indien u dit project zelf wil gebruiken of testen, bezoek dan zeker de git repository\footnote{\url{https://github.ugent.be/gartangh/VOP_Voorraadbeheer}} of de website\footnote{\url{https://github.ugent.be/pages/gartangh/VOP_Voorraadbeheer}}.
Hier kan meer informatie gevonden worden over verscheidene onderwerpen, zoals bijvoorbeeld de installatie, MQTT topics en configuratiebestanden.

\section{Problemen} \label{sec:problems}
Een openstaand probleem van het project is dat het verbinden van de off-board controller met twee verschillende netwerken nog niet optimaal verloopt. Vaak wil de applicatie niet verbinden met het netwerk van de drone.\\

Een tekortkoming aan het project is dat de drone momenteel altijd moet opstarten met de voorkant wijzend naar de positieve x-as.
Dit kan manueel niet altijd even nauwkeurig gebeuren waardoor het tweede algoritme vaker moet corrigeren tijdens een vlucht.
Een illustratie van dit probleem is niet voorhanden, maar het kon wel visueel vastgesteld worden.\\
Een mogelijke oplossing voor dit probleem is om de drone op een bepaald punt in de ruimte vast te zetten zodat de lengte van de drone parallel ligt met de x-as, met de voorkant richting het positieve deel.\\
Een andere oplossing is om de drone eerst te calibreren alvorens de vlucht aan te vatten.
Hierbij kan eerst een aantal meter recht vooruit gevlogen worden en vervolgens bepalen in welke richting de drone gevlogen heeft.
Dit brengt enkele veronderstellingen met zich mee, die haast onmogelijk in werkelijkheid gereproduceerd kunnen worden.
bijvoorbeeld: de drone moet recht vooruit vliegen, zonder af te wijken door luchtstromingen of te roteren tijdens de vlucht.
Ook is er een relatief grote plaats nodig waar de drone in alle mogelijke richtingen enkele meters kan vliegen.\\

\section{Kosten}
Hier komt de verantwoording van gemaakte kosten.\\
De extra batterij van \SI{1000}{\mA\hour} was nodig om op korte tijd voldoende te kunnen testen.
Zonder die extra batterij zou er telkens meer dan een uur gewacht moeten worden na een kwartier testen.\\
De eerste LiPo batterijen (van \SI{150}{\mA\hour}) die aangekocht werden bleken niet lang genoeg stroom te kunnen leveren aan de on-board controller.
Daarna werd geopteerd voor grotere batterijen van \SI{500}{\mA\hour}.\\
De aankoop van de Adafruit HUZZAH bleek overbodig nadat de controller opgesplitst werd. Ook zou het niet meer mogelijk geweest zijn om de gekozen libraries te gebruiken.\\

Een overzicht van alle gemaakte kosten vindt u in tabel \ref{tab:kosten}.\\
Materiaal dat reeds beschikbaar was zoals de Pozyx \textit{location anchors} en \textit{tags} en de Raspberry Pis is niet in de tabel verwerkt.
\begin{table}[p]
\centering
\begin{tabular}{ |l|r|r|r| } \hline
Product & Prijs (\euro{}) & Aantal & Totaal (\euro{}) \\ [.5ex] \hline \hline
Parrot AR.Drone 2.0 Elite Edition & 116.71 & 1 & 116.71 \\ \hline
Micro USB OTG & 1.32 & 2 & 2.63 \\ \hline
5 LiPo batterijen \SI{150}{\mA\hour} en lader voor controller & 12.46 & 1 & 12.46 \\ \hline
Adafruit HUZZAH & 14.57 & 1 & 14.57 \\ \hline
LiPo batterij \SI{1000}{\mA\hour} voor drone & 34.99 & 1 & 34.99 \\ \hline
2 LiPo batterijen \SI{500}{\mA\hour} en lader voor controller & 35.35 & 1 & 35.35 \\ [.5ex] \hline
Velcro & 0.79 & 1 & 0.79 \\ \hline
\hline
Totaal & & & 217.50 \\ \hline
\end{tabular}
\caption[Kosten]{Verantwoording van gemaakte kosten.}
\label{tab:kosten}
\end{table}