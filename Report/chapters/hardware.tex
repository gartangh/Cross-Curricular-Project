\chapter{Hardware}
In wat volgt, worden alle componenten uitvoerig besproken en wordt een verantwoording gegeven van de ontwerpskeuzes.

\section{Drone} \label{sec:drone}
De eerste beslissing omtrent hardware was het uitzoeken van de drone.
Hier werd geopteerd voor de Parrot AR.Drone 2.0 Elite Edition met jungle camo.
De belangrijkste argumenten voor deze keuze zijn de kostprijs (\SI{116.71}{\euro{}}), de vluchtduur (\SI{12}{\min}) en het laadvermogen.\\

De meeste types drone die een lading van zo'n \SI{100}{\g} kunnen dragen minstens \SI{250}{\euro{}}.
Dit type is echter al even niet meer in productie, waardoor de prijs enorm gezakt is.
Een goedkopere drone, die niet onder de minidrones geplaatst wordt, kon niet gevonden worden.\\

Ook voor het software gedeelte is deze drone een goede keuze. Parrot stelt een SDK, die toelaat om de drone via wifi aan te sturen en vlucht gegevens op te halen, openbaar ter beschikking.
Er bestaan reeds libraries, waarvan er in dit project ook gebruik gemaakt wordt.\\
De camera op de drone zou kunnen dienen om barcodes in te scannen, maar dat onderdeel werd niet onder de doelstellingen van dit project gedefini\"eerd.\\
Tot slot bezit de drone ook nog een ultrasone sensor om de hoogte t.o.v. de vloer te meten en een camera om stabiel te blijven zweven op dezelfde positie.

\section{Indoor lokalisatie}  \label{sec:uwb}
In theorie is het mogelijk om, als een drone op een gekende locatie (met een gekende ori\"entatie) vertrekt, deze zonder enige input informatie of correcties een reeks vluchtbewegingen door te geven zodat een voorgeprogrammeerde route gevolgd wordt. In de praktijk wordt dit echter bemoeilijkt door ongekende externe factoren, denk bijvoorbeeld maar aan een ongekend obstakel dat plots het pad van de drone kruist, of een ventilatieschacht die de drone uit positie blaast. Ook het opstijgen gebeurt niet vlekkeloos, waardoor hij met een foutieve ori\"entatie aan zijn tocht zou beginnen. Daarom is het nodig dat het toestel z'n specifieke locatie in de ruimte op elk moment gekend is.\\

Als de drone tussen 2 rekken met een doorgang van \SI{1}{\m} moet kunnen vliegen, dan moet de nauwkeurigheid van lokalisatie in de grootte orde van \SI{0.10}{\m} liggen. Veelgebruikte lokalisatie-technologie\"en zoals gps, wifi en bluetooth zijn te onnauwkeurig voor deze toepassing. Ultra Wide Band komt deze noden tegemoet. Dit is een vrij recente techniek met een nauwkeurigheid in de grootteorde van \SI{0.10}{\m}, wat volstaat om de drone indoor te kunnen lokaliseren.\\

Voor de locatiebepaling werd beroep gedaan op de Pozyx-hardware\footnote{\url{https://www.pozyx.io}}, ontwikkeld door een spin-off van de Ugent, meer bepaald op de anker nodes (die op gekende locaties in de ruimte worden opgehangen) en op een mobiele tag (die als onderdeel van de controller op de drone wordt bevestigd). 
Over de mobiele tag is meer te vinden in sectie \ref{sec:pozyx_tag}.

\section{On-board Controller} \label{sec:onboard_controller}
\subsection{Pozyx tag}  \label{sec:pozyx_tag}
De mobiele tag kan om de beurt de verschillende ankers aanspreken, en opvragen hoe ver hij van hen verwijderd is. Wanneer er enkele van deze afstanden gekend zijn, kan hij zijn locatie bepalen ten opzichte van de ankers.\\

De DecaWave DWM1001 bevat dezelfde UWB chip als de Pozyx tag, maar is goedkoper, compacter, en lichter.
Voor de gebruikte drone is het minieme gewichtsverschil niet echt een probleem.
Men moet echter wel in het achterhoofd houden dat meer massa de stabiliteit en vliegminuten in negatieve zin be\"invloedt.\\
Aangezien de DecaWave veel complexer te programmeren is, werd er voor deze toepassing toch gebruikt gemaakt van de Pozyx tag.

\subsection{Raspberry Pi Zero W} \label{sec:raspberry_pi}
De Raspberry Pi Zero W wordt gebruikt om de Pozyx tag aan te sturen en de link te leggen met de MQTT server.
De verbinding tussen de Pozyx tag en de Raspberry Pi wordt gelegd via een USB OTG kabel, waarvan beide uiteinden van het type micro-USB zijn.

\subsection{LiPo Batterij en Power Supply} \label{sec:lipo}
De Lithium-ion Polymeer Batterij (LiPo) moet de controller gedurende ongeveer een kwartier van stroom kunnen voorzien, aangezien de drone ook ongeveer \SI{15}{\min} lang in de lucht kan blijven.\\
De controller heeft gedurende een kwartier ongeveer \SI{350}{\mA} nodig.
Om pieken op te kunnen vangen werd gekozen voor een batterij van \SI{500}{\mA\hour}.
Deze kan gedurende een kwartier zo'n \SI{2000}{\mA} leveren aan de controller.\\

Omdat de Rasberry Pi Zero W op \SI{5}{\V} opereert i.p.v. op de \SI{3.7}{\V} van de LiPo batterij, wordt er nog een LiPo SHIM tussen geplaatst.
Deze zal niet enkel het voltage omvormen, maar bezit ook een indicator dat inschakeld wanneer de batterij bijna leeg is.

\section{Off-board Controller} \label{sec:offboard_controller}
Om de drone effectief aan te sturen maken we gebruik van een Raspberry Pi 3 B.
Deze heeft de mogelijkheid om met 2 netwerken tegelijk te verbinden. De Ethernet interface wordt gebruikt om verbinding te maken met het internet (en de MQTT server), terwijl de wifi interface gebruikt wordt om met het netwerk van de drone te verbinden.\\

De reden waarom de on en off-board controller opgesplists zijn, is omdat de Rasberry Pi Zero W slechts 1 wifi interface ontersteunt.
Een onderzochte piste is om een Adafruit HUZZAH met een ESP8266 wifi chip te verbinden met de seri\"ele poort van de Raspberry Pi en via deze extra interface met de drone te verbinden.\\
Dit bracht echter problemen met zich mee:
\begin{itemize}
	\item De gebruikte Node.js library zou op low-level aangepast moeten worden.
	\item Het stroomverbruik zou verdubbelen in waarde.
	\item Het gewicht van de controller zou nog meer stijgen.
\end{itemize}

\section{Setup} \label{sec:setup_hardware}
Figuur \ref{fig:setup_hardware} illustreerd de harware setup.
\begin{figure}[p]
	\centering
	\includegraphics[width=\textwidth]{Setup}
	\caption[Hardware setup]{Hardware setup.}
	\label{fig:setup_hardware}
\end{figure}