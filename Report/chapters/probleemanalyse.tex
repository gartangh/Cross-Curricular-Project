\chapter{Analyse van de probleemstelling}
Voorraadbeheer in een magazijn gebeurt traditioneel door werknemers die op een vorklift handmatig barcodes inscannen. Zoals u kan aanvoelen is dat een zeer langzame, dure en gevaarlijke aanpak. Een kosteneffectievere oplossing is het gebruikmaken van automatisch aangestuurde drones die met een camera de barcodes kunnen inscannen.\\

Het doel van dit vakoverschrijdend project is om een basis te leggen voor dit systeem. In eerste instantie is het doel om een drone een vooraf bepaalde route autonoom te laten vliegen in een magazijn.\\
Later kunnen dan uitbreidingen op dit project gemaakt worden. Voorbeelden van mogelijke uitbreidingen zijn: een wifi-mesh van meerdere drones en een functionele RFID scanner die de gescande codes naar het centrale controlepunt verstuurd.