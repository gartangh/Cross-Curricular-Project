\chapter{Analyse van de probleemstelling}
Voorraadbeheer in een magazijn gebeurt traditioneel door werknemers die op een vorklift handmatig barcodes inscannen. Zoals u wel kan aanvoelen is dat een zeer langzame, dure en gevaarlijke aanpak. Een kosteneffectievere oplossing is het gebruikmaken van automatisch aangestuurde drones die met een camera de barcodes kunnen inscannen.\\

Het doel van dit vakoverschrijdend project is om een basis te leggen voor dit systeem. In eerste instantie is het doel om een drone een vooraf bepaalde route autonoom te laten vliegen in een magazijn. In tweede instantie is het doel om het systeem uit te breiden naar meerdere drones die zonder accidenten door elkaar kunnen laten vliegen om zo het proces te versnellen, of grotere magazijnen te onderhouden.