\chapter*{Inleiding}
\addcontentsline{toc}{chapter}{Inleiding}  
Voorraadbeheer in een magazijn gebeurt traditioneel door werknemers die op een vorklift handmatig barcodes inscannen.
Dit is een zeer langzame, dure en gevaarlijke aanpak.
Een kosteneffectievere oplossing maakt gebruik van automatisch aangestuurde drones die met hun camera barcodes kunnen inscannen.\\

Het doel van dit vakoverschrijdend project is om een basis te leggen voor dit systeem, door commercieel beschikbare drones te voorzien van een controller.
Dit is een controlebord, gemonteerd op de drone dat instaat voor aansturing en lokalisatie van de drone en communicatie met een centraal punt.
De geïmplementeerde controller laat toe om de drone autonoom en probleemloos een (vanuit het controlepunt verzonden) route af te leggen.\\

Problemen waarmee rekening moet gehouden worden zijn de nauwkeurigheid van de locatiebepaling, het gewicht van de on-board controller en de kostprijs van het systeem.